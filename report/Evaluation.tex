\section{Evaluation}

\textbf{The Custom Data-Generator}
\newline
A data-generator is conceptually simple. It is a stream of input, through a filter or transformation, to a desired output.
For this generator specifically, the goal was to have it read the matrix of values from the .txt correctly and output it in a way, that the model could interpret. The datagenerator takes the directory as an input, together with specifications of the image-size. The generator makes a list of files and creates a list of labels determined by which directory the file comes from. The generator then needed a function to convert the txt-file to a more image-like file for the model to read. The generator has the function convert,  that takes a batch of files as input. It starts by creating empty arrays to store the values and labels in..  Then it goes through the batch, reads the values from the txt-file, rescales it to normalized values like in the training-data, reshapes the values and appends the values and labels to the empty arrays. It finally returns the batch of values and labels. Through the testing, it turns out, that for now, the generator only works with a batch-size of $1$, but for this purpose it will do. If the generator was supposed to be for the training part, it would need a little more work.  